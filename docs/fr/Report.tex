\documentclass[conference]{IEEEtran}
\IEEEoverridecommandlockouts

% Encoding / French typography (safe for pdflatex)
\usepackage[T1]{fontenc}
\usepackage[utf8]{inputenc}
\usepackage[french]{babel}

\usepackage{cite}
\usepackage{amsmath,amssymb,amsfonts}
\usepackage{algorithmic}
\usepackage{graphicx}
\usepackage{textcomp}
\usepackage{subfig}
\usepackage{xcolor}

\def\BibTeX{{\rm B\kern-.05em{\sc i\kern-.025em b}\kern-.08em
    T\kern-.1667em\lower.7ex\hbox{E}\kern-.125emX}}

\begin{document}

\title{Classification de textes --- Analyse de sentiment (MI201 Projet 3)\\}

\author{\IEEEauthorblockN{1\textsuperscript{er} Carlos Adrian Meneses Gamboa}
\IEEEauthorblockA{\textit{Programme Ing\'enieur en STIC} \\
\textit{ENSTA Paris}\\
Paris, France \\
carlos.meneses@ensta-paris.fr}
\and
\IEEEauthorblockN{2\textsuperscript{e} Jose Daniel Chacon Gomez}
\IEEEauthorblockA{\textit{Programme Ing\'enieur en STIC} \\
\textit{ENSTA Paris}\\
Paris, France \\
jose-daniel.chacon@ensta-paris.fr}
\and
\IEEEauthorblockN{3\textsuperscript{e} Santiago Florido Gomez}
\IEEEauthorblockA{\textit{Programme Ing\'enieur en STIC} \\
\textit{ENSTA Paris}\\
Paris, France \\
santiago.florido@ensta-paris.fr}
}

\maketitle

\begin{abstract}
Ce travail pr\'esente un syst\`eme d'analyse de sentiment pour de courts textes en anglais et compare des m\'ethodes classiques d'apprentissage automatique \`a une approche bas\'ee sur un transformeur.
En utilisant des repr\'esentations standard (sac de mots et TF--IDF), plusieurs classifieurs sont entra\^{\i}n\'es et \'evalu\'es, puis compar\'es \`a un mod\`ele exploitant des embeddings BERT.
Les r\'esultats sont rapport\'es via l'accuracy et le Macro-F1, en mettant en \'evidence des diff\'erences de performance, de robustesse et de co\^ut computationnel.
L'\'etude fournit des rep\`eres pratiques pour choisir une cha\^{\i}ne de classification de sentiment adapt\'ee \`a des contraintes de ressources typiques.
\end{abstract}

\begin{IEEEkeywords}
analyse de sentiment, TAL, classification de textes, TF--IDF, BERT
\end{IEEEkeywords}

\section{Introduction}

L'analyse de sentiment de courts textes devient fondamentale lorsque les perceptions sont consid\'er\'ees comme un actif informationnel critique pour les responsables de produits et de services \cite{b1}. Cela est particuli\`erement pertinent dans le d\'eveloppement de syst\`emes centr\'es sur les \'emotions, capables de fournir des informations exploitables pour am\'eliorer l'exp\'erience utilisateur ou client. Par exemple, ces informations peuvent conduire \`a des ajustements des strat\'egies de support client ou \`a des campagnes marketing plus cibl\'ees \cite{b2}. Dans ce contexte, les r\'eseaux sociaux --- et plus sp\'ecifiquement les messages courts tels que les tweets et les commentaires sur des plateformes multim\'edias --- figurent parmi les sources les plus utilis\'ees pour conduire ce type d'analyse.

Ce projet porte sur l'analyse automatique de sentiment de courts textes en anglais. Une phase exploratoire est d'abord men\'ee : pr\'etraitement du contenu, puis analyse pr\'eliminaire au moyen de m\'ethodes classiques d'apprentissage automatique. Ensuite, la classification est r\'ealis\'ee avec des classifieurs standards (Naive Bayes, r\'egression logistique et SVM lin\'eaire), en utilisant plusieurs sch\'emas de repr\'esentation (sac de mots, TF--IDF au niveau mot, et TF--IDF au niveau caract\`ere). Les performances sont rapport\'ees via l'accuracy, le Macro-F1 et des m\'etriques compl\'ementaires afin d'assurer une comparaison \'equitable.

Dans un second temps, un perceptron multicouche (MLP) entra\^{\i}n\'e sur des repr\'esentations vectoris\'ees est \'evalu\'e, et une alternative bas\'ee sur des embeddings BERT est \'etudi\'ee pour capturer des s\'emantiques contextuelles. Les performances des MLP construits pour chaque famille de repr\'esentation sont compar\'ees sur plusieurs architectures, adapt\'ees \`a la quantit\'e d'information fournie par la vectorisation (ou par BERT) et orient\'ees vers une classification \`a trois classes. La profondeur est contrainte pour limiter le sur-apprentissage, et des couches de dropout sont ins\'er\'ees entre couches cach\'ees afin de r\'egulariser l'entra\^{\i}nement.

Enfin, afin d'am\'eliorer la classification, des strat\'egies bas\'ees sur des grands mod\`eles de langage (LLM) sont \'evalu\'ees en utilisant la version API de Gemma 3-4b-it (Gemini), afin de comparer ses capacit\'es de classification \`a celles des mod\`eles entra\^{\i}n\'es. Par ailleurs, LoRA est utilis\'e pour r\'ealiser un fine-tuning efficace de transformeurs bas\'es sur BERT \cite{b3}.

\section{Q0---Analyse du jeu de donn\'ees avec des mod\`eles classiques d'apprentissage automatique}

Le jeu de donn\'ees \emph{Sentiment Data Analysis} est compos\'e de courts tweets en anglais, class\'es en trois cat\'egories selon leur polarit\'e : positif, neutre ou n\'egatif (colonne \texttt{sentiment}). Des m\'etadonn\'ees sont \'egalement disponibles : le moment de la journ\'ee de publication, l'\^age de l'utilisateur, et son pays d'origine (colonnes \texttt{Age of User} et \texttt{country}).

Un pr\'etraitement du champ texte est d'abord effectu\'e : suppression des valeurs nulles, puis suppression des \emph{stopwords} avec NLTK, afin d'obtenir une colonne de texte trait\'e d\'ebarrass\'ee de mots tr\`es fr\'equents en anglais \`a faible contribution s\'emantique \cite{b4}.

\subsection{Analyse exploratoire (EDA)}

La distribution des classes dans l'ensemble d'entra\^{\i}nement est analys\'ee. Comme indiqu\'e en Fig.~\ref{fig:q0-eda}, la classe neutre est la plus repr\'esent\'ee, mais l'ensemble ne pr\'esente pas de d\'es\'equilibre substantiel susceptible d'induire un biais majeur des classifieurs.

\begin{figure}[!ht]
  \centering
  \includegraphics[width=\linewidth]{images/Q0/Distribution_of_sentiments.png}
  \caption{Distribution des sentiments dans l'ensemble d'entra\^{\i}nement.}
  \label{fig:q0-eda}
\end{figure}

Avant d'appliquer des strat\'egies d'apprentissage automatique, il est utile d'examiner le comportement des termes (sac de mots) vis-\`a-vis des classes de polarit\'e, notamment via une analyse de fr\'equence.

Cela met en \'evidence le r\^ole d'un \emph{vectorizer}, responsable de la conversion d'une collection de textes (telle que la colonne de texte trait\'e) en vecteurs num\'eriques. Le r\'esultat est une repr\'esentation creuse (sparse) par document, appel\'ee \emph{vectorisation}. Selon la construction de cette matrice, plusieurs approches sont possibles. Ici, trois familles sont consid\'er\'ees : BoW (Bag of Words), TF--IDF, et TF--IDF caract\`eres \cite{b5}.

\begin{itemize}
\item \textbf{BoW.} Repr\'esente un document par l'occurrence des mots en n-grammes, en ignorant leur position. Un vocabulaire est construit, puis chaque document est encod\'e par des comptes. Dans scikit-learn, \texttt{CountVectorizer} ``convertit une collection de documents texte en une matrice de comptes de tokens'' \cite{b5}.

\item \textbf{TF--IDF.} Produit une matrice creuse comme BoW, mais pond\`ere les termes par une fr\'equence inverse de document (IDF) afin de p\'enaliser les termes tr\`es fr\'equents \cite{b5}. Le poids TF--IDF d'un terme $t$ dans un document $d$ est donn\'e par l'Eq.~(\ref{eq:tfidf}), avec $\mathrm{tf}(t,d)$ la fr\'equence du terme dans $d$ et $\mathrm{idf}(t)$ une pond\'eration li\'ee \`a sa diffusion dans le corpus.

\begin{equation}
\mathrm{tfidf}(t,d) = \mathrm{tf}(t,d)\times \mathrm{idf}(t).
\label{eq:tfidf}
\end{equation}

En pratique, une version liss\'ee de l'IDF est souvent utilis\'ee, Eq.~(\ref{eq:idf}), o\`u $n$ est le nombre total de documents et $\mathrm{df}(t)$ le nombre de documents contenant $t$.

\begin{equation}
\mathrm{idf}(t)=\log\left(\frac{1+n}{1+\mathrm{df}(t)}\right)+1.
\label{eq:idf}
\end{equation}

Enfin, une normalisation par document stabilise l'\'echelle des caract\'eristiques. Une normalisation $\ell_2$ est consid\'er\'ee, Eq.~(\ref{eq:l2norm}), o\`u $\mathbf{v}$ est le vecteur TF--IDF d'un document.

\begin{equation}
\mathbf{v}_{\mathrm{norm}}=\frac{\mathbf{v}}{\lVert \mathbf{v} \rVert_2}
=\frac{\mathbf{v}}{\sqrt{v_1^2+v_2^2+\cdots+v_n^2}}.
\label{eq:l2norm}
\end{equation}

\item \textbf{TF--IDF caract\`eres.} Applique TF--IDF \`a des n-grammes de caract\`eres plut\^ot que de mots, contr\^ol\'e dans scikit-learn via le param\`etre \texttt{analyzer} \cite{b6}.
\end{itemize}

Apr\`es d\'efinition des repr\'esentations, BoW est utilis\'e pour extraire les 20 uni-grammes et bi-grammes les plus fr\'equents. Pour TF--IDF, l'objectif est d'afficher les 20 uni-grammes et bi-grammes de poids maximal. Les n-grammes de caract\`eres ne sont pas utilis\'es \`a cette \'etape car ils sont moins interpr\'etables pour l'analyse vis\'ee.

\begin{figure}[!ht]
  \centering
  \subfloat[Monogrammes]{
    \includegraphics[width=0.9\linewidth]{images/Q0/top20monogramsBoW.png}
    \label{fig:q0-bow-uni}}
  \hfill
  \subfloat[Bigrams]{
    \includegraphics[width=0.9\linewidth]{images/Q0/top20bigramsBoW.png}
    \label{fig:q0-bow-bi}}
  \caption{Top 20 n-grammes avec BoW.}
  \label{fig:q0-bow}
\end{figure}

\begin{figure}[!ht]
  \centering
  \subfloat[Monogrammes]{
    \includegraphics[width=0.9\linewidth]{images/Q0/top20monogramstidf.png}
    \label{fig:q0-tfidf-uni}}
  \hfill
  \subfloat[Bigrams]{
    \includegraphics[width=0.9\linewidth]{images/Q0/top20bigramstidf.png}
    \label{fig:q0-tfidf-bi}}
  \caption{Top 20 n-grammes avec TF--IDF.}
  \label{fig:q0-tfidf}
\end{figure}

L'analyse des r\'esultats des Figs.~\ref{fig:q0-bow}--\ref{fig:q0-tfidf} montre que certains n-grammes sont coh\'erents avec la polarit\'e associ\'ee (par exemple \emph{bad} ou \emph{sorry} pour la classe n\'egative). N\'eanmoins, ces approches \'etant purement statistiques, des termes fr\'equents \`a faible valeur discriminante (ex. \emph{it}) peuvent aussi appara\^{\i}tre. De plus, certains bigrammes a priori neutres (ex. \emph{feels like}) deviennent repr\'esentatifs en raison du contexte de collecte. Cela anticipe une limite des vectorisations par fr\'equence par rapport \`a des embeddings contextualis\'es.

Il est \'egalement utile de v\'erifier si des variables secondaires (\'age, type de tweet, moment de la journ\'ee) ont un lien avec la polarit\'e. Les distributions (Fig.~\ref{fig:q0-sbuubtt}) restent proches d'une cat\'egorie \`a l'autre ; ces variables ne sont donc pas retenues pour l'entra\^{\i}nement.

\begin{figure}[!ht]
  \centering
  \subfloat[Sentiments par utilisateur]{
    \includegraphics[width=0.9\linewidth]{images/Q0/sentimentsbyuser.png}
    \label{fig:q0-sbu}}
  \hfill
  \subfloat[Sentiments par type de tweet]{
    \includegraphics[width=0.9\linewidth]{images/Q0/sentimentbytypeoftweet.png}
    \label{fig:q0-sbtt}}
  \caption{Distribution comparative des sentiments, stratifi\'ee par utilisateur et type de tweet.}
  \label{fig:q0-sbuubtt}
\end{figure}

Concernant le pays, la Fig.~\ref{fig:q0-dcountry} pr\'esente les pays les plus fr\'equents. Les diff\'erences de distribution restent modestes et de nombreux pays sont faiblement repr\'esent\'es, ce qui peut introduire des caract\'eristiques rares et un risque de sur-apprentissage. Cette colonne n'est donc pas retenue.

\begin{figure}[!ht]
  \centering
  \includegraphics[width=\linewidth]{images/Q0/sentimentbycountry.png}
  \caption{Distribution des sentiments pour les 10 pays les plus repr\'esent\'es.}
  \label{fig:q0-dcountry}
\end{figure}

Enfin, la longueur du texte pr\'etrait\'e est examin\'ee (Fig.~\ref{fig:q0-wordcount}). Les distributions par classe sont comparables ; cette variable n'est pas retenue.

\begin{figure}[!ht]
  \centering
  \includegraphics[width=\linewidth]{images/Q0/wordcountsentiment.png}
  \caption{Distribution du nombre de mots par sentiment.}
  \label{fig:q0-wordcount}
\end{figure}

\subsection{Entra\^{\i}nement de mod\`eles classiques}

Apr\`es l'analyse pr\'ec\'edente, des mod\`eles supervis\'es classiques (hors deep learning) sont entra\^{\i}n\'es pour classer chaque message en trois polarit\'es. Ces mod\`eles s'appuient sur une repr\'esentation explicite obtenue via un vectorizer : repr\'esentation du document, apprentissage du classifieur, \'evaluation \cite{b7}.

Pour chaque sch\'ema de repr\'esentation, quatre classifieurs sont consid\'er\'es : Multinomial Naive Bayes, r\'egression logistique, SVM lin\'eaire, et Random Forest. Une recherche d'hyperparam\`etres est effectu\'ee sur un pipeline \emph{vectorizer + mod\`ele}, \'evalu\'ee via validation crois\'ee stratifi\'ee. La meilleure configuration (selon une m\'etrique cible) est ensuite r\'eentra\^{\i}n\'ee sur l'ensemble complet d'entra\^{\i}nement.

Les hyperparam\`etres principaux des vectorizers sont :
\texttt{ngram\_range} (tailles de n-grammes), \texttt{min\_df} (seuil minimal de fr\'equence documentaire), \texttt{max\_df} (seuil maximal), et \texttt{max\_features} (limite de vocabulaire). Cette limite est particuli\`erement importante pour la TF--IDF caract\`eres afin de contr\^oler la dimension et le co\^ut.

Les hyperparam\`etres des mod\`eles sont sp\'ecifiques et r\'esum\'es ci-dessous :
\begin{itemize}
\item \textbf{Multinomial Naive Bayes} : \texttt{alpha} (lissage additif) pour \'eviter des probabilit\'es nulles \cite{b8}.
\item \textbf{R\'egression logistique} : \texttt{C} (inverse de la r\'egularisation), \texttt{penalty}, \texttt{solver}, \texttt{max\_iter} \cite{b9}.
\item \textbf{SVM lin\'eaire} : \texttt{C}, \texttt{estimator\_\_loss}, \texttt{max\_iter}. Le classifieur est calibr\'e via \texttt{CalibratedClassifierCV} afin d'obtenir des probabilit\'es de classe \cite{b11}.
\item \textbf{Random Forest} : \texttt{n\_estimators}, \texttt{max\_depth}, \texttt{min\_samples\_split}, \texttt{class\_weight} \cite{b12}.
\end{itemize}

Une grille est construite, la performance est estim\'ee via validation crois\'ee \`a 5 plis, puis la meilleure configuration est r\'eentra\^{\i}n\'ee sur l'ensemble complet. Les r\'esultats et leur discussion sont pr\'esent\'es en Q1.

\section{Q1---Classification avec des mod\`eles classiques et analyse de performance}
\label{sec:q1}

Au total, 12 configurations exp\'erimentales sont entra\^{\i}n\'ees (3 sch\'emas de vectorisation $\times$ 4 mod\`eles). L'objectif est de comparer ces combinaisons sur l'ensemble de test et de d\'efinir une base classique solide, servant ensuite de r\'ef\'erence face \`a des m\'ethodes bas\'ees sur des embeddings et des mod\`eles de type BERT.

Les m\'etriques rapport\'ees sont l'Accuracy, le Macro-Recall et le Macro-F1. Les m\'etriques macro sont particuli\`erement pertinentes car elles \'evaluent la performance de mani\`ere plus \'equilibr\'ee entre classes, en r\'eduisant le risque qu'une classe dominante pilote l'\'evaluation globale.

\subsection{Temps d'entra\^{\i}nement}
\label{subsec:q1-time}

Chaque exp\'erience bas\'ee sur \texttt{GridSearchCV} produit un fichier CSV contenant le d\'etail de toutes les combinaisons test\'ees, y compris les hyperparam\`etres du vectorizer et du classifieur. Ces CSV permettent une analyse syst\'ematique des changements de configuration et une s\'election inform\'ee de la meilleure cha\^{\i}ne finale.

Dans ces fichiers, le champ \texttt{mean\_fit\_time} repr\'esente le temps moyen d'entra\^{\i}nement par pli pour chaque configuration \'evalu\'ee. Il permet d'estimer le co\^ut total associ\'e \`a la recherche d'hyperparam\`etres.

\subsubsection{Estimation du temps total (calcul s\'equentiel)}
\label{subsubsec:q1-ttotal}

L'estimation s\'equentielle est calcul\'ee comme la somme des temps moyens par pli, multipli\'ee par le nombre de plis de validation crois\'ee :
\begin{equation}
\label{eq:q1-e1}
T_{\text{total}} \approx \sum_{i=1}^{n_{\text{candidates}}} \left(\texttt{mean\_fit\_time}_i \times n_{\text{splits}}\right),
\qquad
n_{\text{splits}} = 5.
\end{equation}

Ici, $n_{\text{candidates}}$ correspond au nombre de lignes du CSV (nombre de combinaisons \'evalu\'ees) :
\begin{itemize}
    \item $n_{\text{candidates}} = 968$
    \item $T_{\text{total}} \approx 287{,}978.76~\text{s}$
\end{itemize}

\subsubsection{Temps mur avec parall\'elisation}
\label{subsubsec:q1-twall}

La valeur pr\'ec\'edente correspond \`a un temps de calcul th\'eorique s\'equentiel. Comme le parall\'elisme est activ\'e avec \texttt{n\_jobs=-1}, la charge est distribu\'ee sur plusieurs c\oe urs CPU. Un temps mur approximatif peut \^etre estim\'e par :
\begin{equation}
\label{eq:q1-e2}
T_{\text{wall}} \approx \frac{T_{\text{total}}}{n_{\text{jobs}}}.
\end{equation}

En supposant 8 c\oe urs effectifs :
\[
T_{\text{wall}} \approx \frac{287{,}978.76}{8} \approx 35{,}997.35~\text{s} \approx 10.00~\text{h}.
\]

\subsection{Performance sur l'ensemble de test}
\label{subsec:q1-test}

La performance sur test est r\'esum\'ee par les r\'esultats pr\'esent\'es dans les Figs.~\ref{fig:q1-f1}--\ref{fig:q1-f3}. Le Macro-F1, le Macro-Recall et l'Accuracy sont rapport\'es dans les Figs.~\ref{fig:q1-f1}, \ref{fig:q1-f2} et \ref{fig:q1-f3}, respectivement. Pour chaque classifieur, les graphiques comparent les scores obtenus avec chaque m\'ethode de vectorisation, ce qui rend les tendances faciles \`a identifier.

\begin{figure}[!ht]
  \centering
  \includegraphics[width=\linewidth]{images/Q1/barh_models_hue_vectorization_f1_macro.png}
  \caption{Comparaison sur test --- Macro-F1 (couleur = vectorisation).}
  \label{fig:q1-f1}
\end{figure}

\begin{figure}[!ht]
  \centering
  \includegraphics[width=\linewidth]{images/Q1/barh_models_hue_vectorization_recall_macro.png}
  \caption{Comparaison sur test --- Macro-Recall (couleur = vectorisation).}
  \label{fig:q1-f2}
\end{figure}

\begin{figure}[!ht]
  \centering
  \includegraphics[width=\linewidth]{images/Q1/barh_models_hue_vectorization_accuracy.png}
  \caption{Comparaison sur test --- Accuracy (couleur = vectorisation).}
  \label{fig:q1-f3}
\end{figure}

\subsubsection{Impact de la m\'ethode de vectorisation}
\label{subsubsec:q1-vectorization}

Globalement, les figures indiquent que TF--IDF caract\`eres (TF--IDF char) tend \`a fournir les meilleures valeurs sur les trois m\'etriques. Ce comportement est coh\'erent avec la nature des tweets (courts, informels), o\`u la mod\'elisation au niveau caract\`ere capture des signaux sub-lexicaux importants : fautes d'orthographe, variations morphologiques, allongements (ex. ``soooo good''), hashtags, abr\'eviations, et indices pr\'efixe/suffixe.

\subsubsection{Comparaison des mod\`eles et compromis}
\label{subsubsec:q1-tradeoffs}

Parmi les classifieurs, le SVM lin\'eaire et la r\'egression logistique se distinguent par leur constance, en particulier combin\'es \`a TF--IDF (notamment TF--IDF char). Les repr\'esentations BoW/TF--IDF produisant des matrices de tr\`es grande dimension et creuses, les mod\`eles lin\'eaires sont g\'en\'eralement efficaces et peu co\^uteux.

Bien que Random Forest soit tr\`es comp\'etitif (voire meilleur), plusieurs limites sont importantes en haute dimension :
\begin{itemize}
    \item \textbf{Scalabilit\'e et co\^ut :} l'entra\^{\i}nement de nombreux arbres dans un espace TF--IDF augmente fortement le temps et la m\'emoire, surtout sous \texttt{GridSearchCV}.
    \item \textbf{Risque de sur-apprentissage :} TF--IDF char introduit de nombreux n-grammes tr\`es sp\'ecifiques ; un mod\`ele non-lin\'eaire peut capturer des motifs accidentels et d\'egrader la robustesse hors-domaine.
    \item \textbf{Interpr\'etabilit\'e pratique plus faible :} une for\^et est plus difficile \`a justifier qu'un mod\`ele lin\'eaire.
\end{itemize}

Ainsi, m\^eme en cas de gain marginal, les mod\`eles lin\'eaires restent attractifs pour leur stabilit\'e, leur co\^ut et leur clart\'e m\'ethodologique.
\section{Q2---Architectures MLP pour la classification de sentiment de textes courts}

Cette section \'etudie des architectures de perceptron multicouche (MLP) afin d'identifier efficacement la polarit\'e de courts textes en anglais. Le pipeline consid\'er\'e comprend : (i) une repr\'esentation (vectorisation) du texte ; (ii) la construction d'un \texttt{Dataset}/\texttt{DataLoader} ; (iii) l'entra\^{\i}nement d'un MLP ; puis (iv) l'\'evaluation.

Le premier \'el\'ement du pipeline est le vectorizer (BoW, TF--IDF mot, ou TF--IDF caract\`ere). Les hyperparam\`etres retenus pour chaque vectorizer correspondent \`a la meilleure configuration observ\'ee dans la grille associ\'ee \`a ce vectorizer au stade Q1.

Pour BoW et TF--IDF mot, le nombre de caract\'eristiques est de 6689. Pour TF--IDF caract\`eres, le nombre de caract\'eristiques d\'epasse 100{,}000 si aucune limite n'est impos\'ee. Afin de garder une dimension comparable et de rendre l'entra\^{\i}nement tractable, la configuration TF--IDF caract\`eres est conserv\'ee, mais \texttt{max\_features} est fix\'e \`a 10{,}000. Cela permet de comparer des architectures MLP \'equivalentes (m\^emes profondeurs et t\^etes), en ne modifiant que la dimension d'entr\'ee.

La matrice creuse produite par le vectorizer est convertie en format CSR, puis utilis\'ee via un \texttt{SparseBoWDataset} \cite{b13}. Un \texttt{DataLoader} PyTorch it\`ere ensuite sur l'ensemble d'entra\^{\i}nement ; les \'echantillons sont m\'elang\'es \`a chaque \'epoque. La fonction \texttt{collate} densifie les lots et retourne un dictionnaire \{\texttt{"x"}: \texttt{X\_batch}, \texttt{"label"}: \texttt{y\_batch}\} directement utilisable pour le passage avant et le calcul de la perte \cite{b14}.

\subsection{Architectures MLP propos\'ees (repr\'esentations creuses)}
Afin de respecter les contraintes d'entr\'ee et l'objectif de classification \`a trois classes, quatre r\'eseaux de profondeur mod\'er\'ee (3--4 couches cach\'ees) sont propos\'es. La profondeur est limit\'ee car les repr\'esentations BoW/TF--IDF condensent d\'ej\`a une partie de l'information discriminante au niveau des caract\'eristiques ; augmenter excessivement la profondeur tend \`a accro\^{\i}tre la variance et le risque de sur-apprentissage, sans gain proportionnel en g\'en\'eralisation.

Les diff\'erences entre architectures portent principalement sur : (i) des structures en entonnoir (funnel) ; et (ii) le degr\'e de r\'egularisation via dropout.

La taille d'entr\'ee d\'epend du vectorizer : 6689 pour BoW/TF--IDF mot, et 10{,}000 pour TF--IDF caract\`eres. La taille de batch est fix\'ee \`a 128.

% Les descriptions des couches sont laissées identiques à l’original (très techniques),
% afin d’éviter toute perte d’information ou ambiguïté.
\begin{itemize}
\item \textbf{MLP\_1024\_512\_256\_drop0\_3}
\item \textbf{MLP\_2048\_1024\_512\_drop0\_2\_gelu}
\item \textbf{MLP\_1536\_768\_384\_192\_drop0\_25\_SiLU}
\item \textbf{MLP\_4096\_2048\_1024\_drop0\_1\_ReLU}
\end{itemize}

L'entra\^{\i}nement utilise une perte entropie crois\'ee. Chaque r\'eseau est entra\^{\i}n\'e jusqu'\`a 50 \'epoques, avec un crit\`ere d'arr\^et rapide \`a $1\times 10^{-4}$ bas\'e sur la variation de perte entre deux \'epoques cons\'ecutives. L'optimisation est effectu\'ee avec Adam, avec un \texttt{lr} identique entre architectures afin de conserver des conditions comparables.

\subsection{Motivation pour des embeddings contextualis\'es}
Les vectorisations bas\'ees sur des fr\'equences fournissent une repr\'esentation sans information s\'emantique explicite : elles ne capturent que partiellement l'ordre et le contexte. Cela motive l'introduction d'une repr\'esentation dense incorporant le contexte via une tokenisation et un encodage par un mod\`ele de type BERT.

\subsection{Pipeline avec embeddings BERT}
Changer de repr\'esentation implique d'adapter le \texttt{DataLoader} : au lieu de fournir des caract\'eristiques finales, il fournit des tenseurs tokenis\'es, car les embeddings sont produits \`a travers le passage avant de BERT. Dans ce cas, la fonction \texttt{collate} n'est pas requise, car le tokenizer produit des tenseurs de taille compatible. Le \texttt{DataLoader} retourne \texttt{"input\_ids"}, \texttt{"attention\_mask"} et \texttt{"label"} \cite{b15}.

Une fonction d'extraction d'embeddings ex\'ecute BERT en mode \'evaluation, puis applique un pooling pour obtenir un vecteur par texte. Les lots sont concat\'en\'es pour former une matrice dense. Cette matrice est standardis\'ee via \texttt{StandardScaler} (moyenne nulle, variance unit\'e sur train), afin de stabiliser l'entra\^{\i}nement et d'\'eviter qu'une dimension domine par sa variance \cite{b16}.

\subsection{Architectures MLP au-dessus de BERT}
Avec BERT, la dimension d'entr\'ee devient fixe et faible (768 pour BERT\textsubscript{BASE}). Les embeddings sont denses, contrairement aux vecteurs TF--IDF creux. La complexit\'e repr\'esentative est essentiellement port\'ee par l'encodeur (BERT), et la t\^ete de classification peut rester simple ; dans certains cas, une t\^ete lin\'eaire suffit \cite{b17}.

Les architectures consid\'er\'ees sont :
\begin{enumerate}
\item \textbf{LinearHead\_baseline (single linear layer)}
\item \textbf{MLP\_256\_64\_drop0\_2}
\item \textbf{MLP\_128\_32\_drop0\_2}
\item \textbf{MLP\_512\_128\_drop0\_3}
\end{enumerate}

La perte et la proc\'edure d'entra\^{\i}nement restent identiques (entropie crois\'ee, Adam, 50 \'epoques max, arr\^et rapide), afin de conserver une comparaison coh\'erente. Les r\'esultats de l'ensemble des pipelines de cette section sont analys\'es en Q3.

\section{Q3---Analyse comparative des performances MLP et implications pour la s\'election de baseline}
\label{sec:q3}

Cette section propose une analyse interpr\'etative des r\'esultats MLP introduits en Q2. Bien que Q1 se concentre sur des baselines classiques, il est utile d'\'evaluer si l'introduction de non-lin\'earit\'e via des MLP apporte un gain significatif, et quelles familles de repr\'esentations en b\'en\'eficient. L'analyse s'appuie sur le Macro-F1, le Macro-Recall et l'Accuracy sur test, compl\'et\'es par l'inspection des m\'etriques d'entra\^{\i}nement afin de caract\'eriser les \'ecarts train--test.

\subsection{Analyse pr\'eliminaire des MLP (Q2)}
\label{subsec:q3-mlp-prelim}

Les r\'esultats sur test pour Macro-F1, Macro-Recall et Accuracy sont pr\'esent\'es en Fig.~\ref{fig:q3-mlp-test-f1}, Fig.~\ref{fig:q3-mlp-test-recall} et Fig.~\ref{fig:q3-mlp-test-acc}.

\begin{figure}[!ht]
  \centering
  \includegraphics[width=\linewidth]{images/Q3/mlp_test_f1_macro_barh_hue.png}
  \caption{Comparaison MLP sur test --- Macro-F1 (couleur = famille de repr\'esentation).}
  \label{fig:q3-mlp-test-f1}
\end{figure}

\begin{figure}[!ht]
  \centering
  \includegraphics[width=\linewidth]{images/Q3/mlp_test_recall_macro_barh_hue.png}
  \caption{Comparaison MLP sur test --- Macro-Recall (couleur = famille de repr\'esentation).}
  \label{fig:q3-mlp-test-recall}
\end{figure}

\begin{figure}[!ht]
  \centering
  \includegraphics[width=\linewidth]{images/Q3/mlp_test_accuracy_barh_hue.png}
  \caption{Comparaison MLP sur test --- Accuracy (couleur = famille de repr\'esentation).}
  \label{fig:q3-mlp-test-acc}
\end{figure}

Une tendance stable ressort : les configurations bas\'ees sur \textbf{Char TF--IDF} dominent la performance des MLP. En particulier, le MLP Char TF--IDF (4096--2048--1024, dropout 0.10, ReLU) atteint les meilleurs scores (Macro-F1 = 0.711, Macro-Recall = 0.706, Accuracy = 0.708). Une variante (1024--512--256, dropout 0.30, ReLU) reste proche (0.702 / 0.701 / 0.699). Cela sugg\`ere que, pour des tweets, les signaux au niveau caract\`ere (abr\'eviations, variations orthographiques, motifs suffixe/pr\'efixe, langage informel) restent tr\`es informatifs, y compris avec un classifieur non-lin\'eaire.

Les familles \textbf{BoW} et \textbf{TF--IDF mot} constituent un second niveau, avec des meilleurs cas autour de 0.67--0.68. Enfin, les variantes \textbf{BERT} (t\^etes MLP au-dessus d'embeddings fig\'es) restent comp\'etitives mais plus limit\'ees dans ce r\'egime (environ 0.65--0.67), ce qui est coh\'erent avec l'utilisation d'embeddings contextualis\'es sans fine-tuning end-to-end de l'encodeur.

\subsection{Comparaison par familles (qualit\'e pr\'edictive)}
\label{subsec:q3-mlp-families}

La comparaison au niveau des familles est coh\'erente sur les trois m\'etriques : \textbf{Char TF--IDF} produit les MLP les plus performants et de mani\`ere r\'eguli\`ere.

\subsubsection{Char TF--IDF (meilleure famille en performance)}
\label{subsubsec:q3-char-tfidf}

Les MLP Char TF--IDF atteignent les meilleures performances. Le meilleur mod\`ele est Char TF--IDF 4096--2048--1024 (dropout 0.10, ReLU) avec Macro-F1 $\approx 0.711$, Macro-Recall $\approx 0.706$ et Accuracy $\approx 0.708$. Le comportement est coh\'erent avec le domaine tweet : la granularit\'e caract\`ere est robuste aux fautes, abr\'eviations, allongements (ex. ``soooo''), hashtags et variations d'\'ecriture.

\subsubsection{BoW et TF--IDF mot (niveau interm\'ediaire)}
\label{subsubsec:q3-bow-word}

Les MLP BoW et TF--IDF mot sous-performent Char TF--IDF. Le meilleur cas BoW atteint environ Macro-F1 $\approx 0.678$, et le meilleur cas TF--IDF mot environ Macro-F1 $\approx 0.670$. Sur des textes tr\`es courts, la pr\'esence/absence de termes saillants (BoW) peut \^etre aussi utile qu'une pond\'eration de raret\'e au niveau mot, tandis que les repr\'esentations mot restent sensibles au bruit orthographique.

\subsubsection{BERT (comp\'etitif mais non sup\'erieur \`a Char TF--IDF dans ce r\'egime)}
\label{subsubsec:q3-bert}

Pour les mod\`eles BERT, le meilleur r\'esultat est obtenu par l'option la plus simple : BERT + t\^ete lin\'eaire (768$\rightarrow$3), avec Macro-F1 $\approx 0.666$. L'ajout de couches MLP au-dessus des embeddings n'am\'eliore pas les performances et tend \`a les d\'egrader l\'eg\`erement. Une interpr\'etation coh\'erente est que l'espace d'embeddings BERT est d\'ej\`a raisonnablement s\'eparable lin\'eairement, et qu'augmenter la capacit\'e de la t\^ete ajoute des param\`etres qui ne se traduisent pas en meilleure g\'en\'eralisation dans ce r\'egime.

Pour caract\'eriser la g\'en\'eralisation, les m\'etriques d'entra\^{\i}nement sont examin\'ees (Figs.~\ref{fig:q3-mlp-train-acc}--\ref{fig:q3-mlp-train-recall}).

\begin{figure}[!ht]
  \centering
  \includegraphics[width=\linewidth]{images/Q3/mlp_train_accuracy_barh_hue.png}
  \caption{Comparaison MLP sur train --- Accuracy (couleur = famille de repr\'esentation).}
  \label{fig:q3-mlp-train-acc}
\end{figure}

\begin{figure}[!ht]
  \centering
  \includegraphics[width=\linewidth]{images/Q3/mlp_train_f1_macro_barh_hue.png}
  \caption{Comparaison MLP sur train --- Macro-F1 (couleur = famille de repr\'esentation).}
  \label{fig:q3-mlp-train-f1}
\end{figure}

\begin{figure}[!ht]
  \centering
  \includegraphics[width=\linewidth]{images/Q3/mlp_train_recall_macro_barh_hue.png}
  \caption{Comparaison MLP sur train --- Macro-Recall (couleur = famille de repr\'esentation).}
  \label{fig:q3-mlp-train-recall}
\end{figure}

Ces figures mettent en \'evidence un ph\'enom\`ene important : les MLP sur BoW/TF--IDF (mot ou caract\`ere) atteignent souvent des scores tr\`es \'elev\'es sur train (parfois proches de 1.0), m\^eme pour des architectures modestes. Cela indique une forte capacit\'e d'ajustement dans des espaces creux et de grande dimension. Toutefois, ce plafonnement sur train ne garantit pas un gain proportionnel sur test : un \'ecart train--test demeure, ce qui justifie d'interpr\'eter ces r\'esultats comme un compromis (capacit\'e \'elev\'ee mais risque structurel de sur-apprentissage).

Ainsi, le meilleur MLP en performance sur test est Char TF--IDF 4096--2048--1024 (dropout 0.10). Il surpasse la meilleure variante BERT (t\^ete lin\'eaire). N\'eanmoins, du point de vue g\'en\'eralisation, les MLP sur TF--IDF requi\`erent un contr\^ole explicite (r\'egularisation, arr\^et anticip\'e, contr\^ole de capacit\'e effective). En comparaison, BERT + t\^ete lin\'eaire constitue une alternative plus conservatrice : performance plus basse, mais structure plus simple et moins sensible \`a l'augmentation de capacit\'e de la t\^ete.
\section{Q4 : Analyse comparative avec des grands mod\`eles de langage (LLM)}

Cette analyse compare la meilleure architecture bas\'ee sur BERT \`a un grand mod\`ele de langage moderne. L'objectif est d'\'evaluer si un mod\`ele g\'en\'eratif g\'en\'eraliste, utilis\'e uniquement en inf\'erence, peut rivaliser avec un encodeur sp\'ecialis\'e ayant \'et\'e adapt\'e \`a la t\^ache.

\subsection{M\'ethodologie : inf\'erence g\'en\'erative}

Le mod\`ele gemma-3-4b-it est retenu (instruction-tuned, $\sim$4 milliards de param\`etres). Contrairement \`a la baseline BERT de cette partie --- embeddings BERT fig\'es avec une t\^ete lin\'eaire (\texttt{LinearHead(768$\rightarrow$3)}) --- le LLM est \'evalu\'e via \emph{few-shot prompting} avec l'API Google GenAI, sans mise \`a jour de param\`etres.

Le protocole est d\'efini comme suit :
\begin{itemize}
    \item \textbf{Prompt engineering :} le mod\`ele est explicitement instruit \`a se comporter comme un expert en analyse de sentiment.
    \item \textbf{Contexte few-shot :} trois exemples annot\'es (un par classe : Positive, Negative, Neutral) sont fournis avant la requ\^ete cible.
    \item \textbf{Contraintes de sortie :} la g\'en\'eration est contrainte \`a un label mono-mot mapp\'e vers les classes num\'eriques ($0,1,2$).
    \item \textbf{Sous-ensemble d'\'evaluation :} pour des raisons de latence et de quotas API, l'\'evaluation est men\'ee sur 1\,000 exemples repr\'esentatifs du test.
\end{itemize}

\subsection{\'Evaluation de performance}

Les capacit\'es g\'en\'eratives de Gemma sont compar\'ees au mod\`ele \texttt{BERT Linear Head}, qui pr\'esente la meilleure performance parmi les configurations BERT test\'ees pr\'ec\'edemment.

\subsubsection{R\'esultats LLM (Gemma-3-4b-it)}
La Fig.~\ref{fig:gemma_res} pr\'esente le rapport de classification du mod\`ele g\'en\'eratif. Gemma atteint une \textbf{Accuracy de 0.632} et un \textbf{Macro-F1 de 0.630}.

\begin{figure}[!ht]
  \centering
  \includegraphics[width=\linewidth]{images/Q4/gemma_report.png}
  \caption{Rapport de performance pour Gemma-3-4b-it (inf\'erence few-shot). Le mod\`ele pr\'esente une performance mod\'er\'ee et relativement \'equilibr\'ee, avec une confusion notable entre sentiments polaris\'es et classe neutre.}
  \label{fig:gemma_res}
\end{figure}

La matrice de confusion met en \'evidence une faiblesse : une tendance \`a classer des tweets polaris\'es (positifs/n\'egatifs) comme neutres. Ce comportement est souvent associ\'e \`a l'alignement (RLHF) des mod\`eles instruction-tuned, qui peut favoriser des sorties plus neutres en cas d'ambigu\"it\'e.

\subsubsection{R\'esultats bas\'es sur BERT}
La Fig.~\ref{fig:bert_best} pr\'esente la performance de la baseline \texttt{BERT Linear Head}. Ce mod\`ele atteint une \textbf{Accuracy de 0.664} et un \textbf{Macro-F1 de 0.666}.

\begin{figure}[!ht]
  \centering
  \includegraphics[width=\linewidth]{images/Q4/bert_best_report.png}
  \caption{Rapport de performance pour la baseline BERT Linear Head.}
  \label{fig:bert_best}
\end{figure}

Le mod\`ele BERT pr\'esente une diagonale plus marqu\'ee dans la matrice de confusion, ce qui indique un meilleur pouvoir discriminant sur cette distribution de donn\'ees.

\subsection{Discussion : sp\'ecialisation vs g\'en\'eralisation}

La comparaison entre l'encodeur sp\'ecialis\'e (BERT) et le d\'ecodeur g\'en\'eraliste (Gemma) est r\'esum\'ee en Table~\ref{tab:llm_vs_bert}.

\begin{table}[htbp]
\caption{Comparaison directe : encodeur fine-tun\'e vs LLM few-shot}
\begin{center}
\begin{tabular}{|l|c|c|c|}
\hline
\textbf{Mod\`ele} & \textbf{Params} & \textbf{M\'ethode} & \textbf{Macro F1} \\
\hline
\textbf{BERT (lin\'eaire)} & \textbf{110M} & \textbf{Fine-tuning} & \textbf{0.666} \\
\hline
Gemma-3-4b-it & 4B & Few-shot & 0.630 \\
\hline
\end{tabular}
\label{tab:llm_vs_bert}
\end{center}
\end{table}

\begin{enumerate}
    \item \textbf{\'Ecart de performance :} BERT surpasse Gemma d'environ \textbf{3.6\% en Macro-F1}. Cela confirme que, pour ce jeu de donn\'ees, un mod\`ele plus petit adapt\'e au domaine est plus efficace qu'un mod\`ele massif utilis\'e en few-shot.
    \item \textbf{Efficacit\'e de ressources :} l'\'ecart est important. BERT (110M) peut \^etre d\'eploy\'e avec une faible latence sur du mat\'eriel standard, tandis que Gemma (4B) requiert une VRAM significative ou une d\'ependance API.
\end{enumerate}

\section{Q5---Embeddings BERT vs repr\'esentations brutes}

Bien que la diff\'erence entre (i) des repr\'esentations fond\'ees sur des fr\'equences (BoW/TF--IDF) et (ii) des embeddings BERT ait d\'ej\`a \'et\'e abord\'ee en Q2, cette section d\'etaille plus explicitement les diff\'erences conceptuelles entre ces familles.

BERT est un encodeur transformeur pr\'eentra\^{\i}n\'e pour produire des repr\'esentations contextuelles bidirectionnelles : chaque token est encod\'e en tenant compte du contexte \`a gauche et \`a droite \cite{b18}. Apr\`es pooling, un embedding dense de dimension fixe (768) est obtenu par texte, et sert d'entr\'ee au classifieur. Cette r\'eduction de dimension par rapport \`a TF--IDF est un premier facteur distinctif.

Pour quantifier l'effet sur la taille du classifieur, le nombre de param\`etres d'une couche pleinement connect\'ee est rappel\'e en Eq.~(\ref{eq:param_count_fc}) \cite{b19}.

\begin{equation}
params  = (infeatures\cdot outfeatures) + outfeatures
\label{eq:param_count_fc}
\end{equation}

En appliquant cette formule \`a un r\'eseau $\texttt{INPUT\_DIM} \rightarrow 1024 \rightarrow 512 \rightarrow 256 \rightarrow 3$, un total de 10{,}897{,}923 param\`etres est obtenu pour un MLP avec entr\'ee \`a 10{,}000 dimensions (TF--IDF caract\`eres limit\'e), Eq.~(\ref{eq:param_count_mlp_char}), alors que la m\^eme structure avec entr\'ee \`a 768 dimensions (embedding BERT) conduit \`a 1{,}444{,}355 param\`etres, Eq.~(\ref{eq:param_count_mlp_bert}), soit une r\'eduction d'environ 86.75\%. Cette r\'eduction diminue le risque de sur-apprentissage et stabilise l'entra\^{\i}nement, tout en r\'eduisant le co\^ut de calcul pour la t\^ete.

\begin{equation}
\begin{gathered}
P_{1}\,(10000 \rightarrow 1024) = 10000 \cdot 1024 + 1024 = 10{,}241{,}024, \\
P_{2}\,(1024 \rightarrow 512)  = 1024 \cdot 512 + 512 = 524{,}800, \\
P_{3}\,(512 \rightarrow 256)   = 512 \cdot 256 + 256 = 131{,}328, \\
P_{4}\,(256 \rightarrow 3)     = 256 \cdot 3 + 3 = 771, \\
P_{\text{total}}               = P_{1}+P_{2}+P_{3}+P_{4} = 10{,}897{,}923.
\end{gathered}
\label{eq:param_count_mlp_char}
\end{equation}

\begin{equation}
\begin{gathered}
P_{1}\,(768 \rightarrow 1024)  = 768 \cdot 1024 + 1024 = 787{,}456, \\
P_{2}\,(1024 \rightarrow 512)  = 1024 \cdot 512 + 512 = 524{,}800, \\
P_{3}\,(512 \rightarrow 256)   = 512 \cdot 256 + 256 = 131{,}328, \\
P_{4}\,(256 \rightarrow 3)     = 256 \cdot 3 + 3 = 771, \\
P_{\text{total}}               = P_{1}+P_{2}+P_{3}+P_{4} = 1{,}444{,}355.
\end{gathered}
\label{eq:param_count_mlp_bert}
\end{equation}

Les vectorisations par fr\'equences produisent typiquement une matrice creuse : l'information se concentre sur peu d'indices actifs. Cela rend les MLP sensibles \`a la largeur et \`a la r\'egularisation, et complique la s\'election d'architecture, car des variations modestes peuvent accro\^{\i}tre le sur-apprentissage. \`A l'inverse, les embeddings BERT sont denses et de faible dimension, ce qui permet des t\^etes plus simples et des choix de r\'egularisation moins critiques.

Enfin, la diff\'erence la plus marqu\'ee concerne la sensibilit\'e au contexte : BERT encode le contexte via un pr\'eentra\^{\i}nement sur de grands corpus, tandis que BoW/TF--IDF restent des repr\'esentations essentiellement statistiques. Cette propri\'et\'e permet d'explorer des classifieurs plus simples au-dessus d'embeddings BERT (parfois une t\^ete lin\'eaire), car une part importante de la complexit\'e est d\'ej\`a absorb\'ee par l'encodeur.

Il est cependant important de distinguer l'extraction d'embeddings fig\'es du fine-tuning : adapter l'encodeur \`a la t\^ache (par exemple via LoRA, Q7) permet d'aligner les repr\'esentations internes avec la distribution de donn\'ees et l'objectif de classification. Cela att\'enue l'effet \emph{domain-agnostic} d'un extracteur fig\'e et peut am\'eliorer substantiellement la performance.

\section{Q6---Architecture BERT et cadre th\'eorique}

BERT (\emph{Bidirectional Encoder Representations from Transformers}) marque un changement de paradigme en apprentissage de repr\'esentations de langage. Contrairement \`a des mod\`eles unidirectionnels (p.\ ex.\ GPT) ou \`a des concat\'enations peu profondes (p.\ ex.\ ELMo), BERT pr\'eentra\^{\i}ne des repr\'esentations profondes bidirectionnelles en conditionnant simultan\'ement \`a gauche et \`a droite \`a tous les niveaux \cite{b_bert}.

\subsection{Architecture du mod\`ele}

L'architecture est un encodeur Transformer bidirectionnel, bas\'e sur Vaswani et al.\ \cite{b_att}, compos\'e d'une pile de $L$ blocs identiques.

Deux tailles sont classiquement consid\'er\'ees \cite{b_bert} :
\begin{itemize}
    \item \textbf{BERT\textsubscript{BASE} :} $L=12$, taille cach\'ee $H=768$, $A=12$ t\^etes d'attention, $\sim$110M param\`etres.
    \item \textbf{BERT\textsubscript{LARGE} :} $L=24$, $H=1024$, $A=16$, $\sim$340M param\`etres.
\end{itemize}

\begin{figure}[!ht]
  \centering
  \includegraphics[width=\linewidth]{images/Q6/bert_arch_comparison.jpg}
  \caption{Diff\'erences d'architectures de pr\'eentra\^{\i}nement. BERT utilise un Transformer bidirectionnel \cite{b_bert}.}
  \label{fig:bert_arch}
\end{figure}

Chaque bloc comprend (i) une attention multi-t\^etes et (ii) un r\'eseau feed-forward positionnel, avec connexions r\'esiduelles et normalisation de couche.

\subsubsection{Attention multi-t\^etes}
La sortie est donn\'ee par une attention \`a produit scalaire normalis\'e :
\begin{equation}
\mathrm{Attention}(Q, K, V) = \mathrm{softmax}\left(\frac{QK^T}{\sqrt{d_k}}\right)V,
\label{eq:attention}
\end{equation}
o\`u $d_k$ est la dimension des cl\'es, introduite comme facteur d'\'echelle.

\subsubsection{R\'eseau feed-forward et activation GELU}
Le FFN applique deux transformations lin\'eaires avec une non-lin\'earit\'e GELU :
\begin{equation}
\mathrm{FFN}(x) = \mathrm{GELU}(xW_1 + b_1)W_2 + b_2,
\label{eq:ffn}
\end{equation}
avec une dimension interm\'ediaire de 3072 pour \texttt{BERT\textsubscript{BASE}}.

\subsubsection{Connexions r\'esiduelles et normalisation}
\begin{equation}
\mathrm{Output} = \mathrm{LayerNorm}(x + \mathrm{Sublayer}(x)).
\label{eq:addnorm}
\end{equation}

\subsection{Repr\'esentation d'entr\'ee}

\subsubsection{Tokenisation WordPiece}
BERT utilise un vocabulaire WordPiece (30\,000 tokens), limitant l'OOV via des sous-mots (ex.\ ``playing'' $\rightarrow$ \texttt{play} + \texttt{\#\#ing}).

\subsubsection{Somme d'embeddings}
Chaque token est repr\'esent\'e par la somme :
\begin{equation}
\mathbf{E} = \mathbf{E}_{token} + \mathbf{E}_{segment} + \mathbf{E}_{position}.
\label{eq:embeddings}
\end{equation}

\begin{figure}[!ht]
  \centering
  \includegraphics[width=\linewidth]{images/Q6/bert_input_rep.jpg}
  \caption{BERT input representation \cite{b_bert}.}
  \label{fig:bert_input}
\end{figure}

Comme illustr\'e en Fig.~\ref{fig:bert_input}, l'embedding d'entr\'ee de chaque token combine : (i) l'identit\'e lexicale (token), (ii) l'appartenance \`a un segment (A/B), et (iii) la position (ordre). Chaque s\'equence commence par le token \texttt{[CLS]} ; l'\'etat cach\'e final associ\'e \`a \texttt{[CLS]} ($C \in \mathbb{R}^H$) est couramment utilis\'e comme repr\'esentation agr\'eg\'ee pour la classification. Le token \texttt{[SEP]} sert \`a d\'elimitation/s\'eparation des s\'equences.

\subsection{Objectifs de pr\'eentra\^{\i}nement}

BERT est pr\'eentra\^{\i}n\'e sur BooksCorpus et Wikipedia via deux t\^aches : MLM et NSP.

\subsubsection{Masked Language Model (MLM)}
15\% des tokens sont s\'electionn\'es : 80\% remplac\'es par \texttt{[MASK]}, 10\% par un token al\'eatoire, 10\% inchang\'es, afin de r\'eduire le d\'ecalage pr\'eentra\^{\i}nement/fine-tuning.

\subsubsection{Next Sentence Prediction (NSP)}
BERT pr\'edit si une phrase $B$ suit $A$ (IsNext) ou est al\'eatoire (NotNext), afin de mod\'eliser des relations inter-phrases.

\subsection{Strat\'egies d'application}

BERT peut \^etre utilis\'e en fine-tuning end-to-end ou en approche \emph{feature-based} (extraction de caract\'eristiques). La Fig.~\ref{fig:bert_table} illustre des r\'esultats en NER o\`u la concat\'enation des quatre derni\`eres couches est comp\'etitive.

\begin{figure}[!ht]
  \centering
  \includegraphics[width=0.85\linewidth]{images/Q6/bert_ner_table.jpg}
  \caption{Approche feature-based sur CoNLL-2003 NER.}
  \label{fig:bert_table}
\end{figure}

\section{Q7---Fine-tuning avec LoRA pour la classification de sentiment}
\label{sec:q7}

LoRA (\emph{Low-Rank Adaptation}) est une m\'ethode de \emph{Parameter-Efficient Fine-Tuning} (PEFT) permettant d'adapter des mod\`eles pr\'eentra\^{\i}n\'es (p.\ ex.\ Transformers) sans mettre \`a jour l'ensemble des param\`etres. Les poids d'origine sont gel\'es et de petites matrices de rang faible, entra\^{\i}nables, sont inject\'ees dans certaines couches (souvent des projections lin\'eaires de l'attention). Cela r\'eduit fortement le nombre de param\`etres entra\^{\i}nables et le co\^ut d'entra\^{\i}nement/m\'emoire, tout en conservant la capacit\'e de sp\'ecialisation \`a la t\^ache.

\begin{figure}[!ht]
  \centering
  \includegraphics[width=\linewidth]{images/Q7/BERT_+_LoRA_(SEQ_CLS)_Full_Report.png}
  \caption{Rapport d'\'evaluation complet pour \textbf{BERT + LoRA (SEQ\_CLS)} sur test.}
  \label{fig:q7-bert-lora-full-report}
\end{figure}

\subsection{Aper\c{c}u d'impl\'ementation}
\label{subsec:q7-impl}

Le pipeline suit un sch\'ema standard :
(1) pr\'eparation du jeu de donn\'ees et split \emph{train/validation/test} ;
(2) tokenisation via le tokenizer du mod\`ele de base ;
(3) chargement d'un mod\`ele de classification (\texttt{AutoModelForSequenceClassification}) ;
(4) injection des adaptateurs LoRA via \texttt{get\_peft\_model} ;
(5) entra\^{\i}nement avec \texttt{Trainer} et \texttt{TrainingArguments} ;
(6) \'evaluation finale sur test (voir Fig.~\ref{fig:q7-bert-lora-full-report}).

\subsection{Configuration LoRA (justification)}
\label{subsec:q7-lora-params}

La configuration retenue est :
\begin{itemize}
    \item \textbf{\texttt{task\_type=TaskType.SEQ\_CLS}} : PEFT configur\'e pour la classification de s\'equences.
    \item \textbf{\texttt{target\_modules=["query","value"]}} : injection dans des projections influentes (Q/V), bon compromis capacit\'e/co\^ut.
    \item \textbf{\texttt{r=8}} : rang (capacit\'e) mod\'er\'e, adapt\'e \`a l'analyse de sentiment.
    \item \textbf{\texttt{lora\_alpha=16}} : facteur d'\'echelle des mises \`a jour LoRA.
    \item \textbf{\texttt{lora\_dropout=0.1}} : r\'egularisation du chemin LoRA pour limiter le sur-apprentissage.
\end{itemize}

\subsection{R\'esultats sur test : BERT + LoRA (SEQ\_CLS)}
\label{subsec:q7-results}

Le mod\`ele \textbf{BERT + LoRA (SEQ\_CLS)} obtient :
\begin{itemize}
    \item Accuracy = 0.761
    \item Macro-Recall = 0.760
    \item Macro-F1 = 0.763
\end{itemize}
Il rapporte en outre un ROC-AUC pond\'er\'e de 0.901, et des AUC par classe de 0.91 (classe 0), 0.86 (classe 1) et 0.94 (classe 2), ce qui indique une bonne s\'eparabilit\'e globale (Fig.~\ref{fig:q7-bert-lora-full-report}).

\subsection{Comparaison directe avec la baseline embeddings BERT (LinearHead 768$\rightarrow$3)}
\label{subsec:q7-compare-bert-baseline}

Par rapport \`a la baseline pr\'ec\'edente (\emph{Approach = MLP; Representation = BERT; Model = LinearHead(768$\rightarrow$3)}), qui rapporte
Acc = 0.664, Macro-R = 0.662, et Macro-F1 = 0.666,
LoRA am\'eliore de mani\`ere coh\'erente toutes les m\'etriques :
\begin{itemize}
    \item \textbf{Accuracy :} $0.761$ vs.\ $0.664$ $\Rightarrow$ $+0.097$ (environ $+14.6\%$ relatif)
    \item \textbf{Macro-Recall :} $0.760$ vs.\ $0.662$ $\Rightarrow$ $+0.098$ (environ $+14.8\%$ relatif)
    \item \textbf{Macro-F1 :} $0.763$ vs.\ $0.666$ $\Rightarrow$ $+0.097$ (environ $+14.6\%$ relatif)
\end{itemize}

Ces gains sont coh\'erents avec l'objectif de LoRA : au lieu d'entra\^{\i}ner uniquement une t\^ete sur des embeddings fig\'es, l'encodeur est l\'eg\`erement adapt\'e via des mises \`a jour de rang faible dans des modules d'attention, ce qui am\'eliore la repr\'esentation interne pour la classification.

\subsection{Positionnement dans le projet}
\label{subsec:q7-positioning}

Avec Macro-F1 = 0.763, \textbf{BERT + LoRA (SEQ\_CLS)} d\'epasse la baseline BERT + LinearHead ainsi que les meilleurs pipelines classiques/MLP rapport\'es auparavant (cf.\ Table~\ref{tab:q3-classical-vs-mlp}). Ce mod\`ele constitue donc, \`a ce stade, le meilleur mod\`ele entra\^{\i}n\'e du projet.

\section{Conclusion}

Cette section vise à établir un ensemble de conclusions concernant la mise en \oe uvre de méthodes de classification de sentiments pour de courts textes en anglais. Tout d’abord, il est important de souligner l’analyse réalisée durant la phase préliminaire du jeu de données et de ses caractéristiques principales. À partir de cette analyse, on observe que, compte tenu des informations d’entraînement disponibles, la variable d’intérêt principale pour entraîner ce type de classifieur est sans ambiguïté le texte prétraité après suppression des stopwords. En effet, la distribution des classes vis-à-vis des autres variables du jeu de données reste soit constante (comme pour les tranches d’âge), soit peut induire un risque accru de mémorisation de motifs non pertinents, car les éléments de chaque classe au sein de ces catégories ne sont pas représentatifs de l’ensemble de la collecte (comme pour la variable pays).

Concernant l’entraînement des modèles classiques d’apprentissage automatique, les résultats atteignent de bonnes performances, principalement grâce à l’évaluation de multiples pipelines combinant un vectoriseur et un classifieur, ainsi qu’aux variations de leurs grilles d’hyperparamètres. Cela conduit à un total de 956 modèles entraînés, dont les meilleures configurations sont sélectionnées. De plus, parmi les trois types de représentations, les meilleurs résultats sont obtenus avec le SVM, la régression logistique et la forêt aléatoire, avec de légers avantages en F1 et en accuracy pour la forêt aléatoire dans la plupart des cas. Néanmoins, des modèles tels que la régression logistique et le SVM sont plus intelligibles et interprétables. Par conséquent, et considérant que les différences de performance entre la forêt aléatoire et la régression logistique / le SVM ne sont pas suffisamment marquées pour conclure que la forêt aléatoire est la meilleure option en NLP, il est important de noter que, malgré des métriques légèrement supérieures, la forêt aléatoire implique un coût d’entraînement sensiblement plus élevé en temps de calcul et en ressources.

Bien que, dans le cas des modèles classiques, l’importance de la méthode de vectorisation et de son choix soit abordée, l’utilisation d’une recherche d’hyperparamètres permet de limiter cet effet lors de la sélection d’un pipeline, puisque les paramètres du vectoriseur font partie intégrante de la procédure de recherche. En revanche, dans le cas des MLP, le nombre plus réduit de configurations candidates, l’influence de chaque type de vectorisation sur la structure du réseau, ainsi que l’intégration d’embeddings BERT figés avant l’entraînement, conduisent à considérer non seulement le meilleur modèle, mais plus largement le meilleur pipeline, incluant à la fois la représentation et la classification.

Lors de l’analyse des résultats de test des MLP entraînés avec des entrées TF--IDF au niveau des caractères, l’écart entre les performances d’entraînement et de test indique une forte capacité d’ajustement et, par conséquent, un risque élevé de surapprentissage. Des modifications de l’architecture du réseau, principalement en termes de largeur des couches et de régularisation, compte tenu de la matrice creuse compressée en entrée, ne se traduisent pas toujours par des améliorations proportionnelles sur le jeu de test, car le risque de surapprentissage augmente. À l’inverse, l’utilisation d’embeddings BERT figés, bien qu’elle présente des performances attendues avec un nombre de paramètres inférieur à celui des modèles TF--IDF caractères, bénéficie du caractère dense de la matrice d’entrée et d’une complexité nettement réduite du classifieur par rapport aux données d’entraînement. Cela rend les structures plus simples associées à cette représentation particulièrement attractives, comme dans le cas d’une couche de décision unique pour classifier les vecteurs produits par BERT. En effet, pour ce type de représentation, l’augmentation de la profondeur et de la complexité globale du réseau peut être associée à une dégradation des performances sur le jeu de test, précisément parce que, compte tenu des propriétés de BERT et de la densité d’information des embeddings, la structure du réseau et le nombre de paramètres nécessaires peuvent rester limités.

Après l’entraînement des MLP, les modèles classiques, en combinaison avec la vectorisation TF--IDF au niveau des caractères, conservent de meilleures performances que les MLP, y compris ceux utilisant des embeddings BERT, pour deux raisons principales. La première est la rigueur de la recherche d’hyperparamètres réalisée pour les modèles classiques. La seconde est que, pour les modèles basés sur des embeddings BERT, il existe encore une marge d’amélioration, en particulier pour la structure la plus conservatrice qui présente de meilleures performances et un risque réduit de surapprentissage ; cette amélioration est rendue possible par le fine-tuning via LoRA, qui permet d’adapter le processus de représentation au domaine spécifique du jeu de données d’entraînement. Les résultats obtenus montrent que, parmi l’ensemble des modèles évalués, le modèle BERT affiné avec LoRA atteint la meilleure performance en termes de F1 et d’accuracy avec une marge notable ; il constitue donc une structure fortement recommandée pour développer des modèles de classification de courts textes fondés sur la polarité ou le sentiment.


\begin{thebibliography}{00}

\bibitem{b1} T. Finn and A. Downie, ``How can sentiment analysis be used to improve customer experience?,'' \emph{IBM Think}, accessed Jan. 18, 2026. [Online]. Available: https://www.ibm.com/think/insights/how-can-sentiment-analysis-be-used-to-improve-customer-experience

\bibitem{b2} B. Pang and L. Lee, ``Opinion mining and sentiment analysis,'' \emph{Foundations and Trends in Information Retrieval}, vol. 2, no. 1--2, pp. 1--135, 2008.

\bibitem{b3} E. J. Hu, Y. Shen, P. Wallis, Z. Allen-Zhu, Y. Li, S. Wang, L. Wang, and W. Chen, ``LoRA: Low-Rank Adaptation of Large Language Models,'' \emph{arXiv preprint} arXiv:2106.09685, Jun. 2021, doi: 10.48550/arXiv.2106.09685.

\bibitem{b4} The NLTK Project, ``Sample usage for corpus,'' \emph{NLTK Documentation}, accessed Jan. 18, 2026. [Online]. Available: https://www.nltk.org/howto/corpus.html

\bibitem{b5} scikit-learn developers, ``Feature extraction,'' \emph{scikit-learn Documentation}, accessed Jan. 18, 2026. [Online]. Available: https://scikit-learn.org/stable/modules/feature\_extraction.html

\bibitem{b6} scikit-learn developers, ``sklearn.feature\_extraction.text.TfidfVectorizer,'' \emph{scikit-learn Documentation}, accessed Jan. 18, 2026. [Online]. Available: https://scikit-learn.org/stable/modules/generated/sklearn.feature\_extraction.text.TfidfVectorizer.html

\bibitem{b7} F. Sebastiani, ``Machine Learning in Automated Text Categorization,'' \emph{ACM Computing Surveys}, vol. 34, no. 1, pp. 1--47, Mar. 2002, doi: 10.1145/505282.505283. [Online]. Available: https://nmis.isti.cnr.it/sebastiani/Publications/ACMCS02.pdf

\bibitem{b8} scikit-learn developers, ``sklearn.naive\_bayes.MultinomialNB,'' \emph{scikit-learn Documentation}. [Online]. Available: https://scikit-learn.org/stable/modules/generated/sklearn.naive\_bayes.MultinomialNB.html. Accessed: Jan. 18, 2026.

\bibitem{b9} scikit-learn developers, ``sklearn.linear\_model.LogisticRegression,'' \emph{scikit-learn Documentation}. [Online]. Available: https://scikit-learn.org/stable/modules/generated/sklearn.linear\_model.LogisticRegression.html. Accessed: Jan. 18, 2026.

\bibitem{b10} scikit-learn developers, ``sklearn.svm.LinearSVC,'' \emph{scikit-learn Documentation}, accessed Jan. 18, 2026. [Online]. Available: https://scikit-learn.org/stable/modules/generated/sklearn.svm.LinearSVC.html

\bibitem{b11} scikit-learn developers, ``sklearn.calibration.CalibratedClassifierCV,'' \emph{scikit-learn Documentation}, accessed Jan. 18, 2026. [Online]. Available: https://scikit-learn.org/stable/modules/generated/sklearn.calibration.CalibratedClassifierCV.html

\bibitem{b12} scikit-learn developers, ``sklearn.ensemble.RandomForestClassifier,'' \emph{scikit-learn Documentation}, accessed Jan. 18, 2026. [Online]. Available: https://scikit-learn.org/stable/modules/generated/sklearn.ensemble.RandomForestClassifier.html

\bibitem{b13} SciPy developers, ``scipy.sparse.csr\_matrix,'' \emph{SciPy Documentation}. [Online]. Available: https://docs.scipy.org/doc/scipy/reference/generated/scipy.sparse.csr\_matrix.html. Accessed: Jan. 18, 2026.

\bibitem{b14} PyTorch contributors, ``Data Loading and Processing Tutorial,'' \emph{PyTorch Tutorials}. [Online]. Available: https://docs.pytorch.org/tutorials/beginner/data\_loading\_tutorial.html. Accessed: Jan. 18, 2026.

\bibitem{b15} Hugging Face, ``BERT,'' \emph{Transformers Documentation}. [Online]. Available: https://huggingface.co/docs/transformers/en/model\_doc/bert. Accessed: Jan. 18, 2026.

\bibitem{b16} scikit-learn developers, ``sklearn.preprocessing.StandardScaler,'' \emph{scikit-learn Documentation}, accessed Jan. 18, 2026. [Online]. Available: https://scikit-learn.org/stable/modules/generated/sklearn.preprocessing.StandardScaler.html

\bibitem{b17} Sutriawan, Supriadi Rustad, Guruh Fajar Shidik, and Pujiono, ``Performance Evaluation of Text Embedding Models for Ambiguity Classification in Indonesian News Corpus: A Comparative Study of TF-IDF, Word2Vec, FastText, BERT, and GPT,'' \emph{Ing\'enierie des Syst\`emes d'Information}, vol. 30, no. 6, pp. 1469--1482, June 2025, doi: 10.18280/isi.300606. [Online]. Available: https://www.iieta.org/journals/isi/paper/10.18280/isi.300606

\bibitem{b18} J. Devlin, M.-W. Chang, K. Lee, and K. Toutanova, ``BERT: Pre-training of Deep Bidirectional Transformers for Language Understanding,'' \emph{Proceedings of the 2019 Conference of the North American Chapter of the Association for Computational Linguistics: Human Language Technologies, Volume 1 (Long and Short Papers)}, pp. 4171--4186, Minneapolis, Minnesota, Jun. 2019, doi: 10.18653/v1/N19-1423. [Online]. Available: https://aclanthology.org/N19-1423/

\bibitem{b19} PyTorch contributors, ``torch.nn.Linear,'' \emph{PyTorch Documentation}, accessed Jan. 18, 2026. [Online]. Available: https://docs.pytorch.org/docs/stable/generated/torch.nn.Linear.html

\bibitem{b_bert} J. Devlin, M.-W. Chang, K. Lee, and K. Toutanova, ``BERT: Pre-training of Deep Bidirectional Transformers for Language Understanding,'' in \emph{Proceedings of the 2019 Conference of the North American Chapter of the Association for Computational Linguistics: Human Language Technologies}, vol. 1, Minneapolis, Minnesota, Jun. 2019, pp. 4171--4186.

\bibitem{b_att} A. Vaswani et al., ``Attention is all you need,'' in \emph{Advances in Neural Information Processing Systems}, vol. 30, Long Beach, CA, Dec. 2017.

\end{thebibliography}

\end{document}
